\chapter{Introduction} %Luis

TU Wien Space Team is a student organization engaging in various projects in aerospace engineering. Our mission statement is to foster the Know-How and enthusiasm for aerospace technologies in our peers by providing an accessible entry into rocketry and giving members the opportunity to learn. The team is working on a number of different projects ranging from solid propelled two staged rockets, which can reach the edge of space to hydrogen powered autonomous airplanes and from cubesats to rockets with liquid propulsion systems.

The \uH project has its origin as a small scale technology demonstrator for our large rocket concept, Houbolt. The much reduced scale has allowed for faster and cheaper iteration in development and testing and has given us the opportunity to gather experience and work on our testing equipment and ground infrastructure. The goal of the project is to build and launch the first bi-liquid propelled rocket of the Space Team. The design should provide a solid foundation which can be further developed and used in more advanced projects. If regulatory limitations allow it, we would like to make the design open-source after a successful launch at EuRoC.

Our mission objectives are as follows:
\begin{itemize}
    \item Flight to target altitude of \SI{3}{\kilo\meter}
    \item Successful 2-stage recovery of vehicle
    \item Working telemetry and data recording during the whole mission
    \item Secondary mission: test platform for the CubeSat project STS1 from Space Team
\end{itemize}

\section{Academic Program} %Liana
The TU Wien Space Team is a student association with about 250 members. Due to the affiliation to the TU Wien and the technical focus of the projects, the team consists largely of students from the TU Wien and the fields of study Mechanical Engineering, Electrical Engineering, Technical Physics and Computer Science.
However, efforts are being made to make the team more diverse and to include students from other universities and fields of study. As a result, we now have students from e.g. the University of Vienna, the Vienna University of Economics and Business Administration, the University of Applied Sciences St. Pölten, the University of Applied Arts Vienna and University of Applied Sciences Wiener Neustadt in our team.

\section{Stakeholders}
Apart from universities, which gain visibility and reputation through TU Wien Space Team, and students, who have the opportunity to deepen their knowledge and apply it in practice, there are other parties that are essential for the organisation.
Industry sponsors provide resources and mentor the team members, for example by sharing manufacturing techniques or holding workshops. In return, the project provides visibility both internally to highly motivated students with practical experience and with that to possible future employees and externally to potential customers.

\section{Team Structure} %Liana
TU Wien Space Team is divided into eight active projects:
\begin{itemize}
\item The Hound: Two staged solid rocket slated to reach the Karman Line
\item \uH: Bi-liquid propelled rocket
\item SpaceTeamSat1: Fully in-house developed and built cubesat
\item GATE: Large scale liquid propellant engine
\item AcrossAustria: Hydrogen-powered long range autonomous drone
\item Penrose: Hybrid rocket
\item CanSat: Space Team as launch provider for student-built can-sized satellites
\item FIRST: Onboarding project for new Space Team members as an introduction to rocketry
\end{itemize}
In addition, there are organisational and cross-project departments, those being IT, HR, finance and marketing.

\subsection{\uH Team Structure} 
Currently, about 25-30 people are working on project \uH. Although the structures are flat and exchange across the team is encouraged, there are clearly assigned tasks and module leads who are responsible for ensuring that the goals in their respective areas -- these being Recovery, Propulsion, Avionics, Ground Systems, Aerostructure and Software -- are achieved. Additionally, internal module meetings are held weekly to assign and discuss tasks. 

The workload of the team members is rather diverse: while a minimum of 5 hours per week is expected of everyone, active members devote all their available time into the project at stressful phases. Being able to choose the effort so freely allows us to have a diverse team, whereby the willingness to put time and energy into the project is naturally reflected in the weight of the tasks and the position in the team structure.
Apart from that, the team can draw on the experience of former members, who are consulted when questions arise, that have already been dealt with in previous projects.

In addition to the members who work on the rocket directly, there are people within the team who take care of the project's presence on social media, photography and legal matters.

\section{Technical Challenges}
A considerable challenge was to not only build a liquid propulsion system from the ground up, but to also make it as small and lightweight as we did. Eliminating parts is always good for eliminating risks and reducing weight is obviously a positive impact for rocket performance, but that is easier said than done. Additionally we've put a lot of effort into the safety of our design, both by designing systems that are easy to operate and having as little interaction with hardware that is in a potentially harmful state as possible by remote controlling large parts of the ground support equipment.
A compounding factor for all this is that the entire software stack has been designed and developed completely in-house without using external software solutions (excluding CAN driver and the used Operating Systems) - the same goes for mechanical components, which we tried to manufacture in our own workshop safe for the aluminium propellant tanks, some of the plumbing \& valves and the main carbon fiber body tube.

\section{Vision} %Liana
The team's vision is that \uH will become a benchmark for bi-liquid rockets in the field of student rocketry. The entire project should go open source, and thus provide the foundation for further development. The advantages \uH has already had for us -- fast and cost-efficient iteration thanks to small scale and well thought-out designs -- should be of advantage to future rocket enthusiasts.
