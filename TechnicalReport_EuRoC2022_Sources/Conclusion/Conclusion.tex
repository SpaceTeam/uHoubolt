\chapter{Conclusion and Outlook}
\label{sec:conclusion}

Our motivation back when starting the liquid propulsion project within the Space Team was both to be able to work on extremely cool technologies and learn a lot while trying to develop our own rocket engines, as well as pushing the boundaries of what is thought to be possible for a student team to achieve. Throughout the over three years since the first endeavours in liquid engines, we have learned so much about the technology and its limits, developed extensive testing infrastructure and built a staggering amount of prototypes.

The project we have actively worked on has slowly morphed from the gigantic Houbolt rocket to the small scale technology demonstrator \uH and we also branched out into sub-projects like GATE, a Houbolt-sized engine concept to go along with our large engine test stand - but the main goal has stayed the same: pushing boundaries. We have managed to build and operate the largest rocket engine in Austria, built a reliable small scale engine to be used in \uH and now have a rocket to demonstrate these capabilities.

However, all of this would be in vain if all this progress gets lost once the project concludes. No one can be part of such an ambitious project during their free time forever and as such the probably most important part of the project is the final documentation of it. Once all is said and done and the pressure of internal goals and milestones has passed, we plan to clean up and organize our project files; from CAD models to PCB designs, software infrastructure and engine designs. When that is done it is planned to release everything needed to make a "\uH 2.0" to open source - but that's not all! \uH has been a tremendous undertaking, but we do have plans for the future internally as well: Having now proven we can build a reliable engine, it's time to do more interesting things with it than just aimlessly shooting a rocket into the sky. The next twinkle in our eyes is $\mu$Houverer, a craft that can propulsively take off and land powered by a bi-liquid rocket engine.