\chapter{Concept of Operations}
\label{chap:conops}

\section{Rocket lifecycle during EuRoC}
\begin{enumerate}
    \item Rocket gets presented at the exhibition.
    \item On the day before the launch, the rocket gets assembled, the recovery section prepared and the electronics checked.
    \item On the launch day we bring the rocket to the launch site and start going through the launch day checklists.
    \item After landing we recover the rocket and bring it back to the launch.
    \item After inspection of the rocket, we bring it back to the exhibition area.
\end{enumerate}

\section{Launch Procedure}

\subsection{Vehicle Assembly}
\begin{enumerate}
\item Pulling the RBF.
\item Testing electronics - running test and launch sequence.
\item Inserting the RBF halfway.
\item Starting fuel loading process (see chapter ref \ref{Fuel Loading}).
\item Checking the water state (cooling and heating cycle).
\item Checking that the igniter electronics are deactivated.
\item Installing the igniters.
\item Mounting the Fincan to the body tube.
\item Vacating the launch area.
\end{enumerate}

\subsection{Mission Control Setup}
\begin{enumerate}
\item Connecting directed radio link and Raspberry Pi with LoRa shield to server.
\item Powering on server, Mission Control PC and monitors.
\item Connecting Mission Control PC to server via LAN.
\item Opening Mission Control Web-Application on Web-Browser.
\end{enumerate}

\subsection{Launch Pad Setup}
\begin{enumerate}
\item Mounting flame diverter.
\item Connecting GSE to directed radio link.
\item Filling water reservoirs.
\item Flipping, installing and sealing oxidizer bottle inside temperature exchange mantle.
\item From now on continuously check and add ice to cold water reservoir.
\item Installing pressurant bottle.
\item Sliding rocket onto launch rail.
\item Mounting T-nut underneath lower rail button.
\item Securing and closing holddown above lower rail button.
\item Connecting oxidizer, pressurant and electrical umbilicals.
\item Pulling RBF pin halfway to power on avionics.
\item Checking sensors and actuators, verifying movement and calibration, via Mission Control.
\end{enumerate}

\subsection{Fuel Loading} \label{Fuel Loading}
\begin{enumerate}
\item Closing fuel main valve.
\item Mounting the tyre fill adapter connected to the syringe filled with ethanol to the fuel inlet section to open the check valve completely.
\item Applying force to the syringe to fill the tank with a precise amount of fuel and enabling the ullage gas to exit the tank through the small bleed orifice.
\item Removing the syringe from the rocket after filling.
\item Covering the rocket’s fuel inlet section.
\end{enumerate}

\subsection{Final Pad preps}
\begin{enumerate}
\item Opening oxidizer bottle and checking for leaks.
\item Opening pressurant bottle and checking for leaks.
\item Staring pad cameras.
\item Pulling RBF pin to connect power to ignition circuitry.
\end{enumerate}

From this point onwards the rest of the preparations until launch can be done completely remotely. 

\subsection{Oxidizer Loading}
\label{sec:oxidizer_loading}
Pressure and temperature data is closely monitored throughout the whole process.
\begin{enumerate}
\item Closing oxidizer main valve.
\item Activating and setting the vent valve pressure regulation to 30 bar.
\item Closing pressurant vent valve.
\item Opening pressurant tanking valve for filling pressurant until vent valve gets activated. Then closing pressurant tanking valve.
\item Closing oxidizer vent valve.
\item Opening oxidizer fill valve to start the oxidizer loading.
\item Activating the heating cycle in the oxidizer bottle.
\item Quickly closing oxidizer fill valve when plume (venting liquid oxidizer) is visible.
\item Deactivating the heating cycle/activating the cooling cycle in the oxidizer bottle.
\item Opening oxidizer fill valve again after a stabilization of the venting frequency.
\item Quickly closing oxidizer fill valve when plume is visible. 
\item Opening oxidizer vent valve.
\end{enumerate}

\subsection{Pressurant Loading}
\label{sec:pressurant_loading}
\begin{enumerate}
\item Deactivating vent valve pressure regulation.
\item Opening pressurant tanking valve.
\item Waiting for stable pressurization.
\item Closing pressurant tanking valve.
\item Opening pressurant vent valve.
\item Activating umbilical retract of pressurant and oxidizer tanking lines and verifying clean separation.
\end{enumerate}

\subsection{Internal Countdown and Launch}
\begin{enumerate}
\item Activating the rockets internal control via Mission Control after Go/NoGo.
\item The rocket does a continuity check on the igniters, if continuity is given,  start internal countdown.
\item The rocket checks for proper engine performance after ignition. This is measured by a proxy measurement of combustion chamber pressure. The threshold is set beforehand by Mission Control and is usually set to around \SI{10}{\bar}.
\item If proper engine performance is detected by the rocket, it sends a signal to the Launch Pad to release the holddown.
\item Lift-Off is achieved once the electrical umbilical that is magnetically held in place gets disconnected by the rocket moving out of reach.
\end{enumerate}

Until lift-off there is still a possibility for manual abort from Mission Control. Beginning with lift-off and the electrical umbilical disconnecting the rocket is monitoring itself and no manual abort is possible. The rocket is now in powered ascent phase.

The entire engine burn duration is \SI{8}{\second} long, after which the main valves are closed and we have achieved MECO (Main Engine Cut Off). Since the propellant tanks are sized for this \SI{8}{\second} burn duration, alternatively the engine could also just be left open to run out of propellant (which should happen roughly \SI{8.8}{\second} after ignition). This approach has the advantage of not needing to open the main valves again for complete depressurization.

From then on, the rocket is in unpowered ascent until apogee is detected and recovery is triggered.

\subsection{Recovery}
\begin{enumerate}
\item Opening vent valve pressure regulation.
\item Closing vent valve pressure regulation when the tanks are fully depressurized.
\item Opening fuel main valve for remaining fuel unloading.
\item Separation of the nose cone from the body tube at apogee.
\item Drogue chute release at apogee.
\item Main chute release \SI{250}{\meter} altitude. Backup Altimax triggers at \SI{200}{\meter}.
\item Recovering the rocket after landing.
\end{enumerate}

\subsection{GSE Safing}
\begin{enumerate}
\item Stopping all cameras.
\item Closing the oxidizer bottle.
\item Closing the pressurant bottle.
\item Vacating the pad area.
\item Opening oxidizer fill valve.
\item Opening pressurant tanking valve.
\item Checking all pressure data to verify a full depressurization of the system.
\item Announcing "safe state".
\end{enumerate}
