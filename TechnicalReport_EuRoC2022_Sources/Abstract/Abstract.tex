\chapter{Abstract}
\label{chap:abstract}

\uH is a bi-liquid propelled rocket project by the TU Wien Space Team. Designed and manufactured almost exclusively in-house, with the needed Know-How slowly built up over precursor projects and the projects life cycle, the team has managed to create a robust and lightweight rocket together with the necessary Testing and Ground Support Equipment to safely operate it.

The rocket is designed to fly to an altitude of \SI{3}{\kilo\meter} - thus being in the L3 launch category - using its engine powered by ethanol and liquefied nitrous oxide and to then safely return to the ground thanks to a two stage recovery system. The engine design has been iterated upon over years with many improvements on every part from the injector over propellant feed pressures and igniters to combustion chamber concepts. Throughout the entire project special care has been put into safety, ranging from refining checklists to be as clear as possible, over a remote controlled oxidizer loading system to bleed orifices and burst discs for passive depressurization. As liquid rockets need a lot more and more complex Ground Support Equipment than a typical solid propelled rocket, considerable amount of time has also been invested in simple and easy to use Mission Control software that can control both the rocket while on the pad and the GSE.

The payload flown will be the EDU subsystem from the STS1 (SpaceTeamSat1) project from Space Team. This gives the cubesat project the opportunity to flight test parts of their hardware.